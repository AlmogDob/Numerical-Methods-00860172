\documentclass[11pt, a4paper]{article}

\usepackage{amsmath, amssymb, titling}
\usepackage[margin=2.5cm]{geometry}
\usepackage[colorlinks=true, linkcolor=black, urlcolor=black, citecolor=black]{hyperref}
\usepackage{graphicx}
\usepackage{caption}
\usepackage{subcaption}
\usepackage{float}
\usepackage{cancel}
\usepackage{fancyhdr, lastpage}
\usepackage{fourier-orns}
\usepackage{xcolor}
\usepackage{nomencl}
\makenomenclature
\usepackage{etoolbox}
\usepackage{sidecap}
\usepackage{adjustbox}
\usepackage{listings}
\usepackage{matlab-prettifier}
\usepackage[T1]{fontenc}

\sidecaptionvpos{figure}{c}
\setlength{\headheight}{18.2pt}
\setlength{\nomlabelwidth}{1.5cm}

\renewcommand\maketitlehooka{\null\mbox{}\vfill}
\renewcommand\maketitlehookd{\vfill\null}

\renewcommand{\headrule}{\vspace{-5pt}\hrulefill\raisebox{-2.1pt}{\quad\leafleft\decoone\leafright\quad}\hrulefill}
\newcommand{\parder}[2]{\frac{\partial {#1}}{\partial {#2}}}
% \renewcommand\nomgroup[1]{%
%   \item[\bfseries
%   \ifstrequal{#1}{F}{Far--Away Properties}{%
%   \ifstrequal{#1}{N}{Dimensionless Numbers}{%
%   \ifstrequal{#1}{M}{Matrices}{%
%   \ifstrequal{#1}{D}{Diagonals}{%
%   \ifstrequal{#1}{V}{Vectors}{%
%   \ifstrequal{#1}{P}{Dimensionless Average Properties}{}}}}}}
% ]}

\title{Numerical Methods in Aeronautical Engineering \\ HW1 - Theoretical Questions}
\author{Almog Dobrescu ID 214254252}

% \pagestyle{fancy}
\cfoot{Page \thepage\ of \pageref{LastPage}}

\begin{document}

\maketitle
\thispagestyle{empty}
\newpage

\pagenumbering{roman}
% \setcounter{page}{1}

\tableofcontents
\newpage

\pagestyle{fancy}
\pagenumbering{arabic}
\setcounter{page}{1}

\section{Q2}
Given: 
\begin{equation}
    \begin{matrix}
        \displaystyle\frac{dy}{dt}=f_{\left(t,y\right)} &&& \displaystyle y_{\left(0\right)}=1
    \end{matrix}
\end{equation}

\subsection{A}
Since the Euler method is based on Taylor series expansion, let's derive the expansion for $y_{i+1}$:
\begin{equation}
    y_{i+1}=y_i+\left.\frac{h^1}{1!}\frac{dy}{dt}\right|_i+\left.\frac{h^2}{2!}\frac{d^2y}{dt^2}\right|_i+\left.\frac{h^3}{3!}\frac{d^3y}{dt^3}\right|_i+\cdots
\end{equation}
According the Euler method, the local error can be written as:
\begin{equation}
    \begin{matrix}
        \displaystyle y_{i+1}=y_i+\frac{h^1}{1!}f_i+R_i && \colorbox{yellow}{$\displaystyle R_i=\frac{h^2}{2!}\left.\frac{df}{dt}\right|_i\approx\emph{O}\left(h^2\right)$}
    \end{matrix}
\end{equation}

\subsection{B}
Given the function \emph{f}, we can derive \emph{f} and get the supremum of the local error over the hole field:
\begin{equation}
    \colorbox{yellow}{$\displaystyle\max_i{R}=\frac{\displaystyle\max_i{h^2}}{2!}\max_i{\left|\frac{df}{dt}\right|}$}
\end{equation}

\subsection{C}
Assuming \emph{h} is constant, the supremum of the local error is:
\begin{equation*}
    \max_i{R}=\frac{h^2}{2!}\max_i{\left|\frac{df}{dt}\right|}
\end{equation*}
Defining the final time as \emph{T}, the number of integration steps is:
\begin{equation}
    N=\frac{T}{h}
\end{equation}
Hence the supremum of the global error is:
\begin{equation}
    R_\text{global}=\sum_{i=1}^{N}\max_i{R}=\sum_{i=1}^{N}\frac{h^2}{2!}\max_i{\left|\frac{df}{dt}\right|}=\frac{T}{h}\frac{h^2}{2!}\max_i{\left|\frac{df}{dt}\right|}=\frac{h}{2!}\max_i{\left|\frac{df}{dt}\right|}\approx\colorbox{yellow}{$\emph{O}\left(h\right)$}
\end{equation}

\subsection{D}
If we have the equation of \emph{f}, we can derive and find the maximum value of $\displaystyle\frac{df}{dt}$, in the filed.

\newpage
\section{Q3}
Given the ODE:
\begin{equation}
    \begin{matrix}
        \displaystyle\frac{dy}{dx}=-y &&& y_{\left(0\right)}=1
    \end{matrix}
\end{equation}
The analytical solution for the problem is:
\begin{equation}
    \begin{array}{rcl}
        \displaystyle\frac{dy}{y} & = & -dx \\\\
        \displaystyle\int\frac{dy}{y} & = & \int-dx \\\\
        \displaystyle\ln\left(y\right) & = & -x+C \\\\
        & \Downarrow & y_{\left(0\right)}=1 \\\\
        C & = & 0 \\\\
        & \Downarrow &
    \end{array}
\end{equation}
\begin{equation}
    \colorbox{yellow}{$y=\displaystyle e^{-\displaystyle x}$}
\end{equation}

\subsection{A - Forward Differencing}
Given the differencing equation:
\begin{equation}
    \frac{y_{n+1}-y_n}{h}=-y_n\hspace{1cm}\emph{O}\left(h\right)
\end{equation}
and assuming the solution has the following form:
\begin{equation}
    y_n=\alpha\beta^n
\end{equation}
\begin{equation}
    \begin{array}{rcl}
        & \Downarrow & \\\\
        \alpha\beta^{n+1}-\alpha\beta^n & = & -h\alpha\beta^n \\\\
        \beta^{n+1} & = & \left(1-h\right)\beta^n \\\\
        \beta & = & 1-h \\\\
        & \Downarrow & \\\\
        y_n & = & \alpha\left(1-h\right)^n \\\\
        & \Downarrow & y_{\left(0\right)}=1 \\\\
        \alpha & = & 1 \\\\
        & \Downarrow & \\\\
        y_n & = & \left(1-h\right)^n
    \end{array}
\end{equation}
For stability we will demand $y_n$ to approach zero, so:
\begin{equation}
    \begin{array}{rcl}
        0<&1-h&<1 \\\\
        &\colorbox{yellow}{$0<h<1$}&
    \end{array}
\end{equation}

\subsection{B - Backward Differencing}
Given the differencing equation:
\begin{equation}
    \frac{y_{n}-y_{n-1}}{h}=-y_n\hspace{1cm}\emph{O}\left(h\right)
\end{equation}
and assuming the solution has the following form:
\begin{equation}
    y_n=\alpha\beta^n
\end{equation}
\begin{equation}
    \begin{array}{rcl}
        \Downarrow \\\\
        \alpha\beta^n - \alpha\beta^{n-1} & = & -h\alpha\beta^n \\\\
        \beta^{n-1} & = & \left(1+h\right)\beta^n \\\\
        \beta & = & \displaystyle\frac{1}{1+h} \\\\
        & \Downarrow & \\\\
        y_n & = & \alpha\left(\displaystyle\frac{1}{1+h}\right)^n \\\\
        & \Downarrow & y_{\left(0\right)}=1 \\\\
        \alpha & = & 1 \\\\
        & \Downarrow & \\\\
        y_n & = & \left(\displaystyle\frac{1}{1+h}\right)^n
    \end{array}
\end{equation}
For stability we will demand $y_n$ to approach zero, so:
\begin{equation}
    \begin{array}{rcl}
        0<&\displaystyle\frac{1}{1+h}&<1 \\\\
        1&<&h+1 \\\\
        &\colorbox{yellow}{$0<h$}&
    \end{array}
\end{equation}

\subsection{C - Forward Differencing With Averaging}
Given the differencing equation:
\begin{equation}
    \frac{y_{n+1}-y_{n}}{h}=-\frac{1}{2}\left(y_{n+1}+y_n\right)\hspace{1cm}\emph{O}\left(h^2\right)
\end{equation}
and assuming the solution has the following form:
\begin{equation}
    y_n=\alpha\beta^n
\end{equation}
\begin{equation*}
    \Downarrow
\end{equation*}
\begin{equation}
    \begin{array}{rcl}
        \displaystyle\frac{\alpha\beta^{n+1}-\alpha\beta^{n}}{h} & = & \displaystyle-\frac{1}{2}\left(\alpha\beta^{n+1}+\alpha\beta^n\right) \\\\
        \displaystyle\left(1+\frac{h}{2}\right)\beta^{n+1} & = & \displaystyle\left(1-\frac{h}{2}\right)\beta^n \\\\
        \beta & = & \displaystyle\frac{\displaystyle1-\frac{h}{2}}{\displaystyle1+\frac{h}{2}} \\\\
        \beta & = & \displaystyle\frac{2-h}{2+h} \\\\
        & \Downarrow & \\\\
        y_n & = & \alpha\left(\displaystyle\frac{2-h}{2+h}\right)^n \\\\
        & \Downarrow & y_{\left(0\right)}=1 \\\\
        \alpha & = & 1 \\\\
        & \Downarrow & \\\\
        y_n & = & \left(\displaystyle\frac{2-h}{2+h}\right)^n
    \end{array}
\end{equation}
For stability we will demand $y_n$ to approach zero, so:
\begin{equation}
    \begin{array}{rcl}
        0<&\displaystyle\frac{2-h}{2+h}&<1 \\\\
        0<&2-h&<2+h \\\\
        0<&h&<2 \\\\
        &\colorbox{yellow}{$0<h<2$}&
    \end{array}
\end{equation}

\subsection{D - Central Differencing}
Given the differencing equation:
\begin{equation}
    \frac{y_{n+1}-y_{n-1}}{2h}=-y_n\hspace{1cm}\emph{O}\left(h^2\right)
\end{equation}
and assuming the solution has the following form:
\begin{equation}
    y_n=\alpha\beta^n
\end{equation}
\begin{equation*}
    \Downarrow
\end{equation*}
\begin{equation}
    \begin{array}{rcl}
        \displaystyle\frac{\alpha\beta^{n+1}-\alpha\beta^{n-1}}{2h} & = & -\alpha\beta^n \\\\
        \beta^{n+1}+2h\beta^n-\beta^{n-1} & = & 0 \\\\
        \beta^2+2h\beta-1 & = & 0 \\\\
        \beta & = & \displaystyle-h\pm\sqrt{h^2+1} \\\\
        & \Downarrow & \text{h must be positive} \\\\
        \beta & = & \displaystyle-h+\sqrt{h^2+1} \\\\
        & \Downarrow & \\\\
        y_n & = & \alpha\left(\displaystyle-h+\sqrt{h^2+1}\right)^n \\\\
        & \Downarrow & y_{\left(0\right)}=1 \\\\
        \alpha & = & 1 \\\\
        & \Downarrow & \\\\
        y_n & = & \left(\displaystyle-h+\sqrt{h^2+1}\right)^n
    \end{array}
\end{equation}
For stability we will demand $y_n$ to approach zero, so:
\begin{equation}
    \begin{array}{rcl}
        0<&\displaystyle-h+\sqrt{h^2+1}&<1 \\\\
        h<&\displaystyle\sqrt{h^2+1}&<1+h \\\\
        h^2<&h^2+1&<1+2h+h^2 \\\\
        0&<&2h \\\\
        &\colorbox{yellow}{$0<h$}&
    \end{array}
\end{equation}

\subsection{Choosing The Best Method}
I will choose the \emph{Central Differencing Method} since it is the only unconditional stable method with second order local error.

\end{document}