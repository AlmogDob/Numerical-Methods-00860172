\documentclass[11pt, a4paper]{article}
\usepackage{amsmath, amssymb, titling}
\usepackage[margin=2.5cm]{geometry}
\usepackage[colorlinks=true, linkcolor=black, urlcolor=black, citecolor=black]{hyperref}
\usepackage{graphicx}
\usepackage{caption}
\usepackage{subcaption}
\usepackage{float}
\usepackage{cancel}
\usepackage{fancyhdr, lastpage}
\usepackage{fourier-orns}
\usepackage{xcolor}
\usepackage{nomencl}
\makenomenclature
\usepackage{etoolbox}
\usepackage{sidecap}
\usepackage{adjustbox}
\usepackage{listings}
\usepackage{matlab-prettifier}
\usepackage[T1]{fontenc}

\sidecaptionvpos{figure}{c}
\setlength{\headheight}{18.2pt}
\setlength{\nomlabelwidth}{1.5cm}

\renewcommand\maketitlehooka{\null\mbox{}\vfill}
\renewcommand\maketitlehookd{\vfill\null}

\renewcommand{\headrule}{\vspace{-5pt}\hrulefill\raisebox{-2.1pt}{\quad\leafleft\decoone\leafright\quad}\hrulefill}
\newcommand{\parder}[2]{\frac{\partial {#1}}{\partial {#2}}}
% \renewcommand\nomgroup[1]{%
%   \item[\bfseries
%   \ifstrequal{#1}{F}{Far--Away Properties}{%
%   \ifstrequal{#1}{N}{Dimensionless Numbers}{%
%   \ifstrequal{#1}{M}{Matrices}{%
%   \ifstrequal{#1}{D}{Diagonals}{%
%   \ifstrequal{#1}{V}{Vectors}{%
%   \ifstrequal{#1}{P}{Dimensionless Average Properties}{}}}}}}
% ]}

\title{Numerical Methods in Aeronautical Engineering \\ HW3 - Theoretical Questions}
\author{Almog Dobrescu ID 214254252}

% \pagestyle{fancy}
\cfoot{Page \thepage\ of \pageref{LastPage}}

\begin{document}

\maketitle
\thispagestyle{empty}
\newpage

\pagenumbering{roman}
% \setcounter{page}{1}

\tableofcontents
\newpage

\pagestyle{fancy}
\pagenumbering{arabic}
\setcounter{page}{1}

\section{Q2}
The differencing equation is given by:
\begin{equation}
    \begin{array}{rcl}
        \displaystyle \frac{u_{i,j+1}-u_{i,j}}{kG}+\mathcal{O}\left(\frac{k}{G}\right) & = & \displaystyle \alpha\frac{u_{i+1,j}-2u_{i,j}+u_{i-1,j}}{h^2}+\mathcal{O}\left(h^2\right) \\\\
        u_{i,j+1}-u_{i,j} & = & \displaystyle \frac{\alpha kG}{h^2}\left(u_{i+1,j}-2u_{i,j}+u_{i-1,j}\right)+\mathcal{O}\left(\frac{k}{G},h^2\right)
    \end{array}
\end{equation}
Where:
\begin{itemize}
    \item $h=\Delta x$
    \item $k=\Delta y$
    \item $C=\displaystyle\frac{\alpha k}{h^2}$
    \item $G=\displaystyle\frac{1}{2C}\left(1-e^{-2C}\right)=\displaystyle\frac{h^2}{2\alpha k}\left(1-e^{-2\frac{\alpha k}{h^2}}\right)\le1$
\end{itemize}
Hence the truncation error is of the order of $\mathcal{O}\left(\frac{k}{G},h^2\right)$. \\

\noindent We will use von-Neumann Stability Analysis. Let's consider the error:
\begin{equation}
    e_{r,t}=T_{r,t}-\hat{T}_{r,t}
\end{equation}
\begin{itemize}
    \item $T_{r,t}$ - The solution of the differencing equation.
    \item $\hat{T}_{r,t}$ - The solution of the differencing equation with a small noise in the initial conditions.
\end{itemize}
\noindent Let's define:
\begin{equation}
    E_p=T_{p,0}-\hat{T}_{p,0}
\end{equation}
We can rewrite it as:
\begin{equation}
    E_p=\sum_{p=1}^{N}A_pe^{i\beta_n h p},\qquad i=\sqrt{-1},\qquad\beta_n=\frac{n\pi}{N+1}
\end{equation}
We want to check if there is a mode that diverges. Since the equation is linear we only need one mode to diverge to consider the hole scheme as diverged:
\begin{equation}
    E_p=A_pe^{i\beta_n h p}
\end{equation}
We need to check how the error behaves over time and to make sure it diminishes to $E_p$ when $t=0$.
Let's assume the error is of the form of:
\begin{equation}
    E_{p,j}=A_pe^{i\beta h p}\cdot e^{\alpha t_j}
\end{equation}
We are considered stable when $\Re{\left\{\alpha\right\}}<0$. \\
We can rewrite the error:
\begin{equation}
    E_{p,j}=A_pe^{i\beta h p}\cdot e^{\alpha\Delta tj}=A_pe^{i\beta h p}\cdot\xi^j
\end{equation}
Therefore the stability condition is $\left|\xi\right|\le1$ \\

\noindent Since the differencing equation is linear we can demand the error to satisfy the differencing equation:
\begin{equation}
    E_{p,j+1}=\displaystyle E_{p,j}+\frac{\alpha kG}{h^2}\left(E_{p+1,j}-2E_{p,j}+E_{p-1,j}\right)
\end{equation}
After substituting the error we get:
\begin{equation}
    e^{i\beta h p}\cdot\xi^{j+1}=\displaystyle e^{i\beta h p}\cdot\xi^j+\frac{\alpha kG}{h^2}\left(e^{i\beta h \left(p+1\right)}\cdot\xi^j-2e^{i\beta h \left(p\right)}\cdot\xi^j+e^{i\beta h \left(p-1\right)}\cdot\xi^j\right)
\end{equation}
Dividing by $e^{i\beta h p}\xi^j$ we get:
\begin{equation}
    \begin{array}{rcl}
        \xi & = & \displaystyle 1+\frac{\alpha kG}{h^2}\left(e^{i\beta h }-2+e^{-i\beta h }\right) \\\\
        \xi & = & \displaystyle 1+\frac{2\alpha kG}{h^2}\left(\cos\left(\beta h\right)-1\right) \\\\
        \xi & = & \displaystyle 1-\frac{4\alpha kG}{h^2}\sin^2\left(\frac{\beta h}{2}\right)
    \end{array}
\end{equation}
The stability condition is $\left|\xi\right|\le1$ therefore:
\begin{equation}
    \begin{array}{c}
        \left|\displaystyle 1-\frac{4\alpha kG}{h^2}\sin^2\left(\frac{\beta h}{2}\right)\right|\le1 \\\\
        -1\le\displaystyle 1-\frac{4\alpha kG}{h^2}\sin^2\left(\frac{\beta h}{2}\right)\le1
    \end{array}
\end{equation}
The right inequality is always true. Let's check the left inequality:
\begin{equation}
    \begin{array}{c}
        -1\le\displaystyle 1-\frac{4\alpha kG}{h^2}\sin^2\left(\frac{\beta h}{2}\right) \\\\
        2\ge\displaystyle \frac{4\alpha kG}{h^2}\sin^2\left(\frac{\beta h}{2}\right) \\\\
        \displaystyle \frac{1}{2\sin^2\left(\frac{\beta h}{2}\right)}\ge\displaystyle \frac{\alpha kG}{h^2} \\\\
        \displaystyle\frac{\alpha kG}{h^2}\le\frac{1}{2}
    \end{array}
\end{equation}
\subsection{A}
The stability of the method is determined by:
\begin{equation}
    \displaystyle R=\frac{k}{h^2}\le\frac{1}{2\alpha G}
\end{equation}
Hence, the stability limitation of the normal differencing equation is: 
\begin{equation}
    \displaystyle R\le\frac{1}{2\alpha}
\end{equation}
and the stability limitation of the new differencing equation is:
\begin{equation}
    \displaystyle R\le\frac{1}{2\alpha G}\ge\frac{1}{2\alpha}
\end{equation}
So the new differencing equation has a larger range of stable R values. Which means we can use a larger time step and remain stable.
\subsection{B}
The truncation error is of the order of:
\begin{equation}
    \mathcal{O}\left(\frac{k}{G},h^2\right)
\end{equation}
Hence the truncation error of the normal differencing equation is:
\begin{equation}
    \mathcal{O}\left(k,h^2\right)
\end{equation}
and the truncation error of the new differencing equation is:
\begin{equation}
    \mathcal{O}\left(\frac{k}{G}\ge k,h^2\right)
\end{equation}
So the new differencing equation has a larger local precision in time than the normal differencing equation.

\section{Q3}
For the PDE
\begin{equation}
    \parder{U}{t}=\parder{^2U}{x^2}
\end{equation}
The following differencing equation is proposed:
\begin{equation}
    u_{i,j+1}=u_{i,j}+R\left(u_{i-1,j}-u_{i,j}-u_{i,j+1}+u_{i+1,j+1}\right)
\end{equation}
Using von-Neumann Stability Analysis we can write the error as:
\begin{equation}
    E_{p,j}=A_pe^{i\beta h p}\cdot e^{\alpha\Delta tj}=A_pe^{i\beta h p}\cdot\xi^j
\end{equation}
Where:
\begin{itemize}
    \item $i=\sqrt{-1}$
    \item \emph{p} index in space
\end{itemize}
Therefore the stability condition is $\left|\xi\right|\le1$ \\

\noindent Since the differencing equation is linear we can demand the error to satisfy the differencing equation:
\begin{equation}
    E_{p,j+1}=E_{p,j}+R\left(E_{p-1,j}-E_{p,j}-E_{p,j+1}+E_{p+1,j+1}\right)
\end{equation}
After substituting the error we get:
\begin{equation}
    e^{i\beta h p}\cdot\xi^{j+1}=e^{i\beta h p}\cdot\xi^j+R\left(e^{i\beta h \left(p-1\right)}\cdot\xi^j-e^{i\beta h p}\cdot\xi^j-e^{i\beta h p}\cdot\xi^{j+1}+e^{i\beta h \left(p+1\right)}\cdot\xi^{j+1}\right)
\end{equation}
Dividing by $e^{i\beta h p}\xi^j$ we get:
\begin{equation}
    \begin{array}{rcl}
        \xi & = & 1+R\left(e^{-i\beta h}-1-\xi+e^{i\beta h }\cdot\xi\right) \\\\
        \left(1+R-Re^{i\beta h}\right)\xi & = & 1+R\left(e^{-i\beta h}-1\right) \\\\
        \left(1+R-Re^{i\beta h}\right)\xi & = & 1-R+Re^{-i\beta h} \\\\
        \xi & = & \displaystyle \frac{1-R+Re^{-i\beta h}}{1+R-Re^{i\beta h}}
    \end{array}
\end{equation}
The stability condition is $\left|\xi\right|\le1$ therefore:
\begin{equation}
    \begin{array}{c}
        \left|\displaystyle \frac{1-R+Re^{-i\beta h}}{1+R-Re^{i\beta h}}\right|\le1 \\\\
        \left|\displaystyle \frac{1-R+R\left(\cos\left(\beta h\right)-i\sin\left(\beta h\right)\right)}{1+R-R\left(\cos\left(\beta h\right)+i\sin\left(\beta h\right)\right)}\right|\le1 \\\\
        \left|\displaystyle \frac{1-R+R\cos\left(\beta h\right)-iR\sin\left(\beta h\right)}{1+R-R\cos\left(\beta h\right)+iR\sin\left(\beta h\right)}\right|\le1 \\\\
        \displaystyle\sqrt{\frac{\displaystyle\left(\frac{1}{R}-1+\cos\left(\beta h\right)\right)^2+\sin^2\left(\beta h\right)}{\displaystyle\left(\frac{1}{R}+1-\cos\left(\beta h\right)\right)^2+\sin^2\left(\beta h\right)}}\le1 \\\\
        \displaystyle\frac{\displaystyle\left(\frac{1}{R}-1+\cos\left(\beta h\right)\right)^2+\sin^2\left(\beta h\right)}{\displaystyle\left(\frac{1}{R}+1-\cos\left(\beta h\right)\right)^2+\sin^2\left(\beta h\right)}\le1 \\\\
        \displaystyle \left(\frac{1}{R}-1+\cos\left(\beta h\right)\right)^2+\sin^2\left(\beta h\right)\le\displaystyle \left(\frac{1}{R}+1-\cos\left(\beta h\right)\right)^2+\sin^2\left(\beta h\right) \\\\
        \displaystyle \left(\frac{1}{R}-1+\cos\left(\beta h\right)\right)^2\le\displaystyle \left(\frac{1}{R}+1-\cos\left(\beta h\right)\right)^2 \\\\
        \displaystyle \frac{1}{R}-1+\cos\left(\beta h\right)\le\frac{1}{R}+1-\cos\left(\beta h\right) \\\\
        \displaystyle 2\cos\left(\beta h\right)\le2 \\\\
        \displaystyle \cos\left(\beta h\right)\le1
    \end{array}
\end{equation}
Since $\beta h\ge0$ the condition is always meat. So the scheme is unconditionally stable.

\section{Q4}
The following PDE
\begin{equation}
    \parder{U}{t}=\alpha\parder{^2U}{x^2}
\end{equation}
is solve using the Crank-Nicolson method:
\begin{equation}
    u_{i,j+1}=\frac{R}{2}\left(u_{i-1,j+1}-2u_{i,j+1}+u_{i+1,j+1}\right)+\underbrace{text}_{b_i}
\end{equation}

\end{document}